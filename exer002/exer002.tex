\documentclass[12pt]{article}

\usepackage[T1]{fontenc}
\usepackage[utf8]{inputenc}
\usepackage[english]{babel}
\usepackage{csquotes}

\usepackage{amsmath}
\usepackage{amsfonts}
\usepackage{float}
\usepackage{graphicx}
\usepackage[margin=2cm]{geometry}
\usepackage{listings}
\usepackage{siunitx}
\usepackage{subcaption}
\usepackage{tabularx}
\usepackage{titlesec}
\usepackage{tikz}
	\usetikzlibrary{arrows,calc,fit}
\usepackage[colorinlistoftodos, textsize=footnotesize]{todonotes}

\usepackage[colorlinks=true, allcolors=blue, pdfusetitle]{hyperref}
\usepackage[]{cleveref}

% \bibliographystyle{IEEEtran}

% tone down the part command
\makeatletter
\titleformat{\part}[hang]
  {\LARGE\bf\filright}
  {}
  {0mm}
  {}
\makeatother

\lstset{
    tabsize=3,
    breakatwhitespace=true,
    breaklines=true,
    frame=simple
}

\title{Exercise \#2 - ECE1387 - Standard Cell and FPGA Technology Mapping with the ABC Logic Synthesis Framework \& Linear Programming Placement}
\author{Matthew J.P. Walker <matthewjp.walker@mail.utoronto.ca>}
\date{\today}

\begin{document}

\maketitle

\part{FPGA Technology Mapping}
\section{FPGA Technology Mapping for Logic Depth}
%\todo{for each circuit: number of LUTs, and their max depth}
\begin{table}[H]\centering\begin{tabular}{ r | *2c }
\hline\hline
Circuit & \#LUTs & Max LUT depth \\
\hline
\texttt{alu4}    &  823 & 6 \\
\texttt{apex2}   & 1042 & 6 \\
\texttt{ex1010}  & 2944 & 6 \\
\texttt{misex3}  &  817 & 5 \\
\texttt{pdc}     & 2397 & 7 \\
\hline\hline
\end{tabular}\caption{}\label{tab:fpga-map-for-depth}\end{table}

\section{FPGA Technology Mapping for Logic Area}
%\todo{for each circuit: amount of area reduction, depth difference}
%\todo{average area \& depth redution}
\begin{table}[H]\centering\begin{tabular}{ r | c | r l || c | r l }
\hline\hline
Circuit &
	\#LUTs &
	\multicolumn{2}{p{3.5cm} || }{\raggedright\#LUT Difference from \cref{tab:fpga-map-for-depth}.} &
	\multicolumn{1}{p{1.85cm} | }{\raggedright Max LUT depth} &
	\multicolumn{2}{p{5cm}      }{\raggedright Max LUT Depth Difference from \cref{tab:fpga-map-for-depth}.} \\
\hline
\texttt{alu4}   &  806 &  -17 & (-2.1\%) &  9 & +3 & (+50\%) \\
\texttt{apex2}  &  962 &  -80 & (-7.7\%) & 11 & +5 & (+83\%) \\
\texttt{ex1010} & 2835 & -109 & (-3.7\%) & 11 & +5 & (+83\%) \\
\texttt{misex3} &  776 &  -41 & (-5.0\%) &  9 & +4 & (+80\%) \\
\texttt{pdc}    & 2317 &  -80 & (-3.3\%) & 12 & +5 & (+71\%) \\
\hline
\textbf{Average} & & & \textbf{\ -4.4\%} & & & \textbf{\ +74\%} \\
\hline\hline
\end{tabular}\caption{}\label{tab:fpga-map-for-min-area}\end{table}

\part{Standard Cell Technology Mapping}
\section{Unmodified \texttt{mcnc.genlib}}
%\todo{for each circuit: area, delay \& No. cells on crit path}
\begin{table}[H]\centering\begin{tabular}{ r | *3c }
\hline\hline
Circuit & Area & \(T_{min}\) & Logic Depth on Crit. Path \\
\hline
\texttt{alu4}   &  4573.00 & 12.90 & 11 \\
\texttt{apex2}  &  5276.00 & 14.40 & 13 \\
\texttt{ex1010} & 14292.00 & 13.80 & 11 \\
\texttt{misex3} &  4265.00 & 12.00 & 11 \\
\texttt{pdc}    & 12944.00 & 15.20 & 13 \\
\hline\hline
\end{tabular}\caption{}\label{tab:sc-map-on-full-lib}\end{table}

\section{Minimal \texttt{mcnc.genlib}}
%\todo{for each circuit: area, delay \& No. cells on crit path}
\begin{table}[H]\centering\begin{tabular}{ r | *3c }
\hline\hline
Circuit & Area & \(T_{min}\) & Logic Depth on Crit. Path \\
\hline
\texttt{alu4}   &  6006.00 & 23.80 & 25 \\
\texttt{apex2}  &  6888.00 & 27.60 & 29 \\
\texttt{ex1010} & 19881.00 & 25.70 & 27 \\
\texttt{misex3} &  5646.00 & 21.90 & 23 \\
\texttt{pdc}    & 17010.00 & 29.50 & 31 \\
\hline\hline
\end{tabular}\caption{}\label{tab:sc-map-on-minimal-lib}\end{table}

\part{NAND-LUT Equivalency}
\section{}
% \todo{for each circuit: No. LUTs in both earlier FPGA mappings, NAND-equiv number, NAND-equiv to LUT ratio}
%\todo{average nand-equiv}
\begin{table}[H]\centering\begin{tabular}{ r *3c || l  r | c }
\hline\hline
Circuit & \#NAND-2 & \#INV1 & Nand-Equiv & Optim'ion & \#LUTs & \({\text{Nand-Equiv}}/{\text{\#LUT}}\) \\
\hline
\texttt{alu4}   & 2365 & 1276 & 3003.0 & delay &  823 &  3.6 \\
                &      &      &        & area  &  806 &  3.7 \\
\texttt{apex2}  & 2610 & 1668 & 3444.0 & delay & 1042 &  3.3 \\
                &      &      &        & area  &  962 &  3.6 \\
\texttt{ex1010} & 7269 & 5343 & 9940.5 & delay & 2944 &  3.4 \\
                &      &      &        & area  & 2835 &  3.5 \\
\texttt{misex3} & 2166 & 1314 & 2823.0 & delay &  817 &  3.5 \\
                &      &      &        & area  &  776 &  3.6 \\
\texttt{pdc}    & 6627 & 3756 & 8505.0 & delay & 2397 &  3.5 \\
                &      &      &        & area  & 2317 &  3.7 \\
\hline
\textbf{Average}&      &      &        &       &      & \textbf{3.5} \\
\hline\hline
\end{tabular}\caption{}\label{tab:nand-equiv}\end{table}

\newcommand{\placementpicture}[3]{
	\begin{tikzpicture}[
		net1line/.style={red,thick},
		net2line/.style={blue,thick},
		net3line/.style={green,thick},
		blockport/.style={shape=circle,draw,inner sep=0.5mm}
	]
	\coordinate (A1) at (#1);
	\coordinate (B1) at (#2);
	\coordinate (C1) at (#3);
	\coordinate (A2) at ($ ( (1,1) + (A1) $);
	\coordinate (B2) at ($ ( (1,2) + (B1) $);
	\coordinate (C2) at ($ ( (2,1) + (C1) $);

	\draw[step=1,black] (0,0) grid (4,4);
	\node[fit={(A1) (A2)},draw,fill=gray,inner sep=0] (A) {}; \node at (A.base) {A};
	\node[fit={(B1) (B2)},draw,fill=gray,inner sep=0] (B) {}; \node at (B.base) {B};
	\node[fit={(C1) (C2)},draw,fill=gray,inner sep=0] (C) {}; \node at (C.base) {C};

	\foreach \blockports [count=\net] in
		{{A/north,C/west},
		 {A/east,B/west,C/south},
		 {B/east,C/east}}
	{
		\foreach \blockid/\direction in \blockports {
			\node[blockport,net\net line,anchor=\direction] (\blockid Port\direction) at (\blockid.\direction) {};
		}
		\def\portN{test}
		\foreach[count=\ii] \blockid/\direction in \blockports {
			\global\edef\portN{\blockid Port\direction}
		}

		\foreach \blockid/\direction [remember={{\blockid Port\direction}} as \prevport (initially \portN)] in \blockports {
			\draw[net\net line] (\prevport) -- (\blockid Port\direction);
		}
	}

	\end{tikzpicture}
}

\part{Lineal Programming Placement}
\section{}
%\todo{LP file}
%\todo{final placed positions (lower left corners)}
\vspace{-0.75cm}
\begin{figure}[H]\centering
\subcaptionbox{
	A solution with \(x_i,y_i \in \mathbb{R}\). \(cost = 3.5\). \\
	\((x_A,y_A) = (0.0, 1.0)\) \\
	\((x_B,y_B) = (1.5, 0.0)\) \\
	\((x_C,y_C) = (0.5, 2.0)\) \\
}{
	\placementpicture{0,1}{1.5,0}{0.5,2}
}
\hspace{1cm}
\subcaptionbox{
	A solution with \(x_i,y_i \in \mathbb{Z}\). \(cost = 3.5\). \\
	\((x_A,y_A) = (1, 1)\) \\
	\((x_B,y_B) = (2, 0)\) \\
	\((x_C,y_C) = (1, 2)\) \\
}{
	\placementpicture{1,1}{2,0}{1,2}
}
\end{figure}

The LP file used follows

\lstinputlisting{part6-4x4chip.lp}

\end{document}
