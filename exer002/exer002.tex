\documentclass[12pt]{article}

\usepackage[T1]{fontenc}
\usepackage[utf8]{inputenc}
\usepackage[english]{babel}
\usepackage{csquotes}

\usepackage{amsmath}
\usepackage{amsfonts}
\usepackage{float}
\usepackage{graphicx}
\usepackage[margin=2cm]{geometry}
\usepackage{listings}
\usepackage{siunitx}
\usepackage{tabularx}
\usepackage{titlesec}
\usepackage[colorinlistoftodos, textsize=footnotesize]{todonotes}

\usepackage[colorlinks=true, allcolors=blue, pdfusetitle]{hyperref}
\usepackage[]{cleveref}

% \bibliographystyle{IEEEtran}

% tone down the part command
\makeatletter
\titleformat{\part}[hang]
  {\LARGE\bf\filright}
  {}
  {0mm}
  {}
\makeatother

\lstset{
    tabsize=3,
    breakatwhitespace=true,
    breaklines=true,
    frame=simple
}

\title{Exercise \#2 - ECE1387 - Standard Cell and FPGA Technology Mapping with the ABC Logic Synthesis Framework \& Linear Programming Placement}
\author{Matthew J.P. Walker <matthewjp.walker@mail.utoronto.ca>}
\date{\today}

\begin{document}

\maketitle

\part{FPGA Technology Mapping}
\section{FPGA Technology Mapping for Logic Depth}
\todo{for each circuit: number of LUTs, and their max depth}

\section{FPGA Technology Mapping for Logic Area}
\todo{for each circuit: amount of area reduction, depth difference}
\todo{average area \& depth redution}

\part{Standard Cell Technology Mapping}
\section{}
\todo{for each circuit: area, delay \& No. cells on crit path}

\section{minimal gate lib}
\todo{for each circuit: area, delay \& No. cells on crit path}

\part{NAND-LUT Equivalency}
\section{}
\todo{for each circuit: No. LUTs in both earlier FPGA mappings, NAND-equiv number, NAND-equiv to LUT ratio}

\part{Lineal Programming Placement}
\section{}
\todo{LP file}
\todo{final placed positions (lower left corners)}

\end{document}
